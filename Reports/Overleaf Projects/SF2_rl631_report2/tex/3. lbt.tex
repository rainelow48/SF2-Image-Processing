The LBT transforms overlapping blocks in the original image to generate a smaller non-overlapping block in the transformed image. A type of LBT involves a pre-filtering operation (Photo Overlap Transform (POT)) with a filter \texttt{Pf}, followed by a DCT. Compression is achieved by the quantisation of the transformed image at this step. The original image can be recovered by an inverse DCT, followed by a post-filtering operation with a filter \texttt{Pr}. The pre-filtering operation reduces the correlation between blocks, and hence reduces the block artefacts when applied together with the DCT, allowing a smoother and visually pleasing image.\\

The scaling factor $s$ weights the relative contributions of \texttt{Pf} and \texttt{Pr}, and determines the degree of bi-orthogonality. An LBT with POT scaling factors ranging from $1 < s < 2$ is implemented with a $8 \times 8$ DCT on the lighthouse image, with the step size adjusted to match the reference RMS error at $4.86$. Compression ratios for the respective scaling factors are plotted in Figure \ref{fig:LBT scaling}. It can be observed that the maximum compression ratio occurs around $s = 1.4$ at $3.13$, with a quantisation step size of $25.885$. Reconstructed images for $s = 1.1$, $1.4$ and $1.7$ is shown in Figure \ref{fig:LBT s range}, and it is observed that there are no striking differences between the images. The bases and pre-filtered images \texttt{Xp} with different scaling factors are shown in Figure \ref{fig:LBT bases and Xp}. It can be observed from $s=1.1$ and $s=1.7$ that as $s$ increases, the pre-filtered image has more block artefacts, to reduce correlation between adjacent blocks. However, since the reconstructed images are visually similar, an optimal scaling factor should be chosen to maximize compression ratio, $s=1.4$.\\

Using a scaling factor of $1.4$, LBT is performed with $4 \times 4$, $8 \times 8$ and $16 \times 16$ blocks, with quantisation step sizes chosen as $28.755$, $25.885$ and $22.919$ to match the reference RMS error. The respective compression ratios\footnote{\label{footnote:N=16}The compression ratio is obtained by calculating the total bits using $16 \times 16$ blocks across all $N$.} are $3.56$, $3.42$ and $2.95$, and the reconstructed images are shown in Figure \ref{fig:LBT 4 8 16}. Compared to the DCT images in Figure \ref{fig:DCT 4 8 16}, the block artefacts are significantly reduced, with an improved compression ratio$^\ref{footnote:N=16}$ as compared to DCT alone. While the block artefacts are significantly reduced in the N-point LBT, it is still fairly obvious when $N = 4$, but not as much when $N=8$ or $16$, thus an optimal choice of N is 8.

